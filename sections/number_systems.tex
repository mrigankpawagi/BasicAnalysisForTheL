\section{Number Systems}

\subsection{Natural Numbers}

\subsubsection{Definition}
\begin{true}
A relation $R$ on $A$ is a \textbf{partial order} if it satisfies
\begin{itemize}
    \item Reflexivity, i.e., $xRy \ \forall x \in A$
    \item Anti-symmetry, i.e., $xRy \land yRx \implies x = y \ \forall x,y \in A$
    \item Transitivity, i.e., $xRy \land yRz \implies xRz \ \forall x,y,z \in A$
\end{itemize}
Additionally, if $R$ satisfies
\begin{itemize}
    \item $xRy$ or $yRx \ \forall x,y \in A$
\end{itemize}
then $R$ is an \textbf{order} (or total order) on $A$. A set with a partial order is called a partially ordered set or \textbf{poset}, and a set with a total order is called an ordered set or \textbf{totally ordered set}.
\end{true}
\begin{forthel}
[set/sets] [element/elements]
[belong/belongs]

Signature SetSort.
    A set is a notion.

Signature ElmSort.
    Let S be a set.
    An element of S is a notion.

Let x belongs to y stand for (x is an element of y).
Let x is in y stand for (x belongs to y).
\end{forthel}
