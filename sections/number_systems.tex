\section{Number Systems}

\subsection{Natural Numbers}

\subsubsection{Definition}
% TODO: Add definition of natural numbers

\begin{true}

\end{true}
\begin{forthel}
Definition DefNaturalNumbers.
    ...

Let N denote (the set of natural numbers).
\end{forthel}

\subsection{Relations}

\subsubsection{Definition}
\begin{true}
A relation $R$ on $A$ is a \textbf{partial order} if it satisfies
\begin{itemize}
    \item Reflexivity, i.e., $xRy \ \forall x \in A$
    \item Anti-symmetry, i.e., $xRy \land yRx \implies x = y \ \forall x,y \in A$
    \item Transitivity, i.e., $xRy \land yRz \implies xRz \ \forall x,y,z \in A$
\end{itemize}
Additionally, if $R$ satisfies
\begin{itemize}
    \item $xRy$ or $yRx \ \forall x,y \in A$
\end{itemize}
then $R$ is an \textbf{order} (or total order) on $A$. A set with a partial order is called a partially ordered set or \textbf{poset}, and a set with a total order is called an ordered set or \textbf{totally ordered set}.
\end{true}
\begin{forthel}
Definition DefReflexiveRelationOn. 
    Let S be a set.
    A reflexive relation on S is a relation R on S such that
    xRx for every (x << S).

Definition DefAntiSymmetricRelationOn. 
    Let S be a set.
    An antisymmetric relation on S is a relation R on S such that
    (xRy and yRx => x = y) for every (x << S, y << S).

Definition DefTransitiveRelationOn.
    Let S be a set.
    A transitive relation on S is a relation R on S such that
    (xRy and yRz => xRz) for every (x << S, y << S, z << S).

Definition DefPartialOrderOn.
    Let A be a set.
    A partial order on A is a relation R on A such that
    R is reflexive and R is antisymmetric and R is transitive.

Definition DefOrderOn.
    Let A be a set.
    An order on A is a relation R on A such that
    (R is a partial order on A) and ((xRy or yRx) for every (x << A, y << A)).

Let R is a total order on A denote (R is an order on A).

Definition DefPartiallyOrderedSet
    A partially ordered set is a pair (A,R) such that A is a set and R is a partial order on A.

Let (A,R) is a poset denote ((A,R) is a partially ordered set).

Definition DefTotallyOrderedSet.
    A totally ordered set is a pair (A,R) such that A is a set and R is a total order on A.

Let (A,R) is an ordered set denote ((A,R) is a totally ordered set).
\end{forthel}

\subsubsection{Definition}
\begin{true}
An \textbf{equivalence relation} on a set $A$ is a relation $R$ satisfying
\begin{itemize}
    \item Reflexivity
    \item Symmetry, i.e., $xRy \implies yRx \ \forall x,y \in A$
    \item Transitivity
\end{itemize}
\end{true}
\begin{forthel}
Definition DefSymmetricRelationOn. 
    Let S be a set.
    A symmetric relation on S is a relation R on S such that
    (xRy => yRx) for every (x << S, y << S).

Definition DefEquivalenceRelationOn.
    Let S be a set.
    An equivalence relation on S is a relation R on S such that
    R is reflexive and R is symmetric and R is transitive.
\end{forthel}

\subsubsection{Definition}
\begin{true}
An \textbf{equivalence class} of an element $x$ of a set $A$ with respect to an equivalence relation $R$ on $A$ is defined as the set
\begin{align*}
    [x]_R = \{y \in A \mid xRy\}
\end{align*}
\end{true}
\begin{forthel}
Definition DefEquivalenceClassWithRespectTo.
    Let S be a set.
    Let R be an equivalence relation on S.
    Let x be an element of S.
    The equivalence class of x with respect to R is the set E
    such that for every (y << S) (y is an element of E <=> xRy).

Let [x](R) denote (the equivalence class of x with respect to R).
\end{forthel}

\subsection{Integers}

\subsubsection{Definition}
\begin{true}
The set of integers is defined as
\begin{align*}
    \mathbb{Z} = \{ [x]_R \mid x \in \mathbb{N} \times \mathbb{N} \}
\end{align*}
where $R$ is the equivalence relation on $\mathbb{N} \times \mathbb{N}$ defined by
\begin{align*}
    (a,b)R(c,d) \iff a + d = b + c
\end{align*}
\end{true}
\begin{forthel}
Definition DefIntegers.
    Let R be the relation on NxN such that
    for every (a << NxN, b << NxN, c << NxN, d << NxN)
    ((a,b)R(c,d) <=> a + d = b + c).
    The set of integers is the set Z such that
    for every x (x is an element of Z <=> (for some (a << NxN) x = [a](R))).

Let Z denote (the set of integers).
\end{forthel}

\subsection{Rational Numbers}
